\documentclass[a4paper, 11pt, DIV=9]{scrartcl}

\usepackage[french]{babel}
\usepackage[autostyle, french=guillemets]{csquotes}

\usepackage[protrusion=true]{microtype}

\linespread{1.03}

\usepackage{minted}

\usepackage{fontspec}
\setmainfont{STIX Two Text}
\setsansfont{Roboto Medium}
\setmonofont[Scale=.89]{Iosevka Term}

\title{Projet de programmation orientée objet \\ et d'interfaces graphiques}
\author{Julien \textsc{Coolen} \and Yaniv \textsc{Benichou}}
\date{Année 2018 - 3\textsuperscript{e} semestre L2}

\begin{document}

\maketitle

\section{Introduction}

Nous abordons dans ce rapport nos choix de modélisation et
les difficultés rencontrées au cours de ce projet pour le cours de
programmation orientée objet et d’interfaces graphiques.

Le but de ce projet consiste à implémenter plusieurs jeux de
plateau en regroupant les fonctionnalités communes. Pour cela
nous avons développé une plateforme logicielle (en anglais \textit{framework}) à partir
de laquelle nous avons codé chaque jeu: un ensemble d'objets qui intéragissent
entre eux et permettent de développer des jeux de plateau dans la plus grande
généralité. Ce framework devait être facilement étendu pour implémenter un
nouveau jeu.

Nous avons codé tous les jeux demandés: dominos que l'on assemble pour former
une chaîne, dominos gomettes que l'on assemble pour former une chaîne avec des
branches, un puzzle ainsi que le jeu du saboteur.

%Tip : Penser à comment tu veux agencer tes pièces, quelles structures
%de données ça implique, ce que tu veux pouvoir faire, dans quelles
%conditions, quel impact ça aurait sur le reste, etc.

\section{Modélisation}

Nous avons tout d'abord remarqué que chaque jeu s'articule toujours autours des mêmes objets:
un plateau, des pièces, des cartes, une pioche, des joueurs et leur main.
Ce sont les composant de base de notre framework. Chacun de ces objets est
éventuellement étendu pour ajouter ou redéfinir des comportements.


\section{Difficultés rencontrées}

Nous avons développé le projet itérativement en travaillant à plusieurs niveaux
d'abstraction: sur chaque jeu et le framework. Le développement du framework
était guidé par les besoins de chaque jeu et réciproquement le développement
d'un jeu dépendait directement de l'organisation du framework. Ainsi, si une
extension du code était compliquée à écrire, nous revoyions le modèle. 
Cet aspect du projet à été de loin le plus intéressant et enrichissant.

Lors du développement nous avons réécrit à plusieurs reprises certaines
méthodes, ce qui introduisait quelques fois de nouveaux bugs. C'est pourquoi
nous avons écrit des tests unitaires à l'aide de la librairie jUnit que l'on lance à
chaque modification pour identifier toute régression dans le code.

Nous nous sommes également servi du gestionnaire de version git pour sauvegarder
des états du projet au fur et à mesure et collaborer à deux.
\section{Pistes d'amélioration}
\section{Conclusion}

\end{document}

%%% Local Variables: 
%%% coding: utf-8
%%% mode: latex
%%% TeX-engine: luatex
%%% End: 
