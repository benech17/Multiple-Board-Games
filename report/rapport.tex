\documentclass[a4paper, 11pt, DIV=9]{scrartcl}

\usepackage[french]{babel}
\usepackage[autostyle, french=guillemets]{csquotes}

\usepackage[protrusion=true]{microtype}

\linespread{1.03}

\usepackage{minted}

\usepackage{fontspec}
\setmainfont{STIX Two Text}
\setsansfont{Roboto Medium}
\setmonofont[Scale=.89]{Iosevka Term}

\title{Projet de programmation orientée objet \\ et d'interfaces graphiques}
\author{Julien \textsc{Coolen} \and Yaniv \textsc{Benichou}}
\date{Année 2018 - 3\textsuperscript{e} semestre L2}

\begin{document}

\maketitle

\section{Introduction}

Nous abordons dans ce rapport nos choix de modélisation et
les difficultés rencontrées au cours de ce projet pour le cours de
programmation orientée objet et d’interfaces graphiques.

Le but de ce projet consiste à implémenter plusieurs jeux de
plateau en regroupant les fonctionnalités communes. Pour cela
nous avons développé une plateforme logicielle (en anglais \textit{framework}) à partir
de laquelle nous avons codé chaque jeu.

%Tip : Penser à comment tu veux agencer tes pièces, quelles structures
%de données ça implique, ce que tu veux pouvoir faire, dans quelles
%conditions, quel impact ça aurait sur le reste, etc.

\section{Modélisation}

Nous avons remarqué que chaque jeu utilise toujours les mêmes composants:
un plateau, une pièce, une pioche, des joueurs et leur main.

\section{Difficultés rencontrées}


Lors du développement nous avons réécrit à plusieurs reprises certaines
méthodes, ce qui introduisait quelques fois de nouveaux bugs. C'est pourquoi
nous avons écrit des tests unitaires à l'aide de la librairie jUnit que l'on lance à
chaque modification pour identifier toute régression dans le code.

Plus git pour sauvegarder des états de l'avancement en cas de problème
\section{Pistes d'amélioration}
\section{Conclusion}

\end{document}

%%% Local Variables: 
%%% coding: utf-8
%%% mode: latex
%%% TeX-engine: luatex
%%% End: 
